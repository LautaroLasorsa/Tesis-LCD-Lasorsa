\chapter{Datos reales: Mejora de la capacidad predictiva}

En este cápitulo exploramos si las modificaciones propuestas en la medición del bienestar económico mejoran la capacidad de predicir otros indicadores socio económicos.

\section{Esperanza de Vida al Nacer (EVN)}
f
Se estudia pormenorizadamente el caso de la EVN, discutiendo distintas formas de evaluar la capacidad predictiva de las distintas mediciones. Se utiliza el año como variable de control para evitar que una correlación entre el bienestar economico y el año y entre el año y la EVN aumente artificialmente la correlación entre el bienestar economico y la EVN.


\section{Otros indicadores}

Utitizando la metrica comparativa finalmente desarrollada para el caso de la EVN, se aplica la misma metrica a otros indicadores socio economicos (ej: Alfabetización a los 15 años) para ver cómo se comportan es estos casos.