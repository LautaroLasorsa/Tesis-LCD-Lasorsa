%\documentclass[11pt,a4paper,twoside]{tesis}
% SI NO PENSAS IMPRIMIRLO EN FORMATO LIBRO PODES USAR
\documentclass[11pt,a4paper]{tesis}

\usepackage{graphicx}
\usepackage[utf8]{inputenc}
\usepackage[spanish]{babel}
\usepackage[left=3cm,right=3cm,bottom=3.5cm,top=3.5cm]{geometry}

\begin{document}

%%%% CARATULA

\def\autor{Lautaro Lasorsa}
\def\tituloTesis{Propuesta de mejora en la medición del bienestar económico}
\def\runtitulo{Propuesta de mejora en la medición del bienestar económico}
\def\runtitle{Proporsal of improvement in the measurement of economic wellfare}
\def\director{Rodrigo Castro}
\def\codirector{Walter Sosa Escudero}
\def\lugar{Buenos Aires, Argentina, 2024}
\input{caratula}

%%%% ABSTRACTS, AGRADECIMIENTOS Y DEDICATORIA
\frontmatter
\pagestyle{empty}
%\begin{center}
%\large \bf \runtitulo
%\end{center}
%\vspace{1cm}
\chapter*{\runtitulo}

El uso del logaritmo del ingreso promedio como indicador del bienestar económico de una sociedad tiene diversos problemas. El presente trabajo propone un nuevo indicador, el promedio de los logaritmos de los ingresos, como mejor medición del bienestar económico. Comprobamos que, si los ingresos se distribuyen de forma log-normal, está puede ser aproximada con las encuestas de ingresos ya existentes. A su vez, comparamos si resulta en un mejor predictor de otras variables relacionadas al bienestar económico que la metodología original y que otras mediciones alternativas, corregir por desigualdad o tomar el logaritmo de la mediana, llegando a la conclusión de que estas mediciones alternativas se comportan de forma indistinguible a la medición propuesta en este trabajo, y que a su vez existe una cantidad significativa de casos donde el mejor predictor es el logaritmo del promedio de los ingresos. Esto se comprobó tanto para encuestas de ingresos como para encuestas de consumo (usando el consumo como proxy del ingreso)

\bigskip


\noindent\textbf{Palabras claves:} Ingresos, Bienestar económico, Medición, Correlación, Esperanza de Vida, Consumo.

\cleardoublepage
%\begin{center}
%\large \bf \runtitle
%\end{center}
%\vspace{1cm}
\chapter*{\runtitle}

The use of the logarithm of the average income as an indicator of the economic well-being of a society has several problems. This work proposes a new indicator, the average of logarithms of income, as a better measurement of economic well-being. We verified that, if income is distributed in a log-normal manner, this can be approximated with existing income surveys. At the same time, we compare whether it results in a better predictor of other variables related to economic well-being than the original methodology and other alternative measurements, correcting for inequality or taking the logarithm of the median, arriving at the conclusion that these alternative measurements behave in a way that is indistinguishable from the measurement proposed in this work, and that in turn there is a significant number of cases where the best predictor is the logarithm of the average income. This was verified for both income surveys and consumption surveys (using consumption as a proxy for income).
\bigskip

\noindent\textbf{Keywords:} Income, Economic well-being, Measurement, Correlation, Life Expectancy, Consumption. % OPCIONAL: comentar si no se quiere

\cleardoublepage
\chapter*{Agradecimientos}

\begin{itemize}
    \item \textbf{A mis padres:} Por haber hecho posible que yo llegue hasta este punto, por haberme apoyado en todo momento y por colaborar con la corrección de este trabajo.
    \item \textbf{A Maria, mi prometida:} Por su acompañamiento a lo largo de estos últimos años.
    \item \textbf{A mis directores, Rodrigo y Walter:} Por su predisposición y sus valiosos consejos para la realización de este trabajo.
    \item \textbf{A todos los docentes que he tenido a lo largo de mi vida:} Por lo que me han enseñado y su contribución a quién soy hoy.
\end{itemize}
 % OPCIONAL: comentar si no se quiere

\cleardoublepage
\hfill \textit{A quien quiera leer esto, puesto que un escrito no tiene valor salvo en la medida que es leído.}
  % OPCIONAL: comentar si no se quiere

\cleardoublepage
\tableofcontents

\mainmatter
\pagestyle{headings}

%%%% ACA VA EL CONTENIDO DE LA TESIS

\chapter{Medición del Bienestar Ecónomico}

En este cápitulo se presenta la problematica a tratar en la tésis, se motiva su estudio y se comparán las formas actuales de medirlo con las alternativas propuestas.

\section{Definición y motivación}

\subsection{Definición}

En este trabajo definiremos el bienestar económico (BE) de un individuo como la utilidad que obtiene de sus ingresos. Es decir, el valor que puede obtener de los recursos economicos, principalmente monetarios, de los que dispone.

Sea $W$ el nivel de ingresos de un individuo, modelamos la utilidad que obtiene de esos ingresos como $U = log(W)$. Entonces, $BE = U(W) = log(W)$

Es importante tener presente que si bien al definirlo como $BE = log(W)$ estamos dando una definición constructiva que nos permite calcularlo a partir de los datos recabados, esta es una aproximación que realizamos al concepto en base al comportamiento que esperamos que tenga:

\begin{itemize}
    \item \textbf{Es creciente}: Esperamos que un mayor ingreso cause un mayor bienestar económico en el inviduo que lo recibe. Formalmente, $W_1 \geq W_2 \rightarrow U(W_1) \geq U(W_2)$
    \item \textbf{Ley de utilidades marginales decrecientes}: Esperamos que, a partir de un valor $W_1$, la utilidad marginal del ingreso sea decreciente. Formalmente:
    $$
    \exists W_1 / \forall W \geq W_1, \epsilon > 0, U(W+\epsilon)-U(W) > U(W+2*\epsilon)-U(W+\epsilon)
    $$
\end{itemize}

Este es un comportamiento analogo a la especificación y la implementación de una función en programación, donde las expectativas que tenemos sobre el indicador (su definición conceptual) cumplen el rol de la especificación y la definición objetivamente medible cumple el rol de la implementación.

\subsection{Motivación}

La correcta medición del bienestar económico tiene diversas motivaciones, como por ejemplo:

\begin{itemize}
    \item Es un objeto de los objetos de estudio elementales de las ciencias económicas.
    \item Permite medir el impacto de las politicas publicas.
    \item Es un insumo para otras disciplinas y puede utilizarse como predictora de otras variables de interes.
\end{itemize}

De estas motivaciones puede deducirse que tiene tanto un valor intrinseco (por si mismo) como un valor instrumental, y que en conjunto vuelven a la correcta medición de este concepto.

Notar que al ser un concepto al cuál aproximamos con una definición constructiva, hay 2 aspectos independientes:

\begin{itemize}
    \item La calidad de la implementación que realicemos, es decir, qué tan bien captura la definición objetivamente medible las expectativas que tenemos sobre el concepto.
    \item La calidad de la medición de esta implementación.
\end{itemize}

De estos dos puntos, el presente trabajo propone posibles mejoras sobre el segundo basandose en la definición empirica ya explicada. Sin embargo, al estudiar la correlación entre el bienestar económico y otras variables, ambos puntos tendrán impacto aunque solo el segundo se estudie explicitamente.

\section{Medición actual}

\subsection{Metodología}

La medición actual del $BE$, utilizada por ejemplo por el \textbf{Índice de Desarrollo Humano}, es mediante el logaritmo del promedio de los íngresos.

En el caso puntual del IDH utiliza como insumo el GNI (Gross Domestic Income, Ingreso Nacional Bruto) a Paridad de Poder Adquisitivo (PPA, PPP en inglés). Es decir:

$$
IDH_{income} = log(GNI\ per\ cápita)
$$

Generalizandolo, dada una población de $N$ individuos, cuyos ingresos son $X_1, X_2, \dots, X_N$, la medición actual del bienestar ecónomico es:

$$
    BE = log(\frac{1}{N} * \sum_{i=1}^{N}X_i)
$$

\subsection{Ventajas}

La principal ventaja de este metodo es a nivel logistico, dado que calcula el $BE$ en base a una colección de datos que pueden ser calculados o estimados de forma independiente:

\begin{itemize}
    \item \textbf{GNI PPP}: El ingreso total de los individuos del país, deflactado a la paridad de poder adquisitivo. Inclusive puede ser posible calcular de forma separada el GNI nomial y el deflactor de PPP, pero también hay metodologías que apuntan directamente al valor GNI PPP utilizando cantidades en vez del valor monterio de los bienes y servicios. 
    \item \textbf{Población}: La población de esa economía durante ese año. Usualmente se utiliza la cantidad de población a mitad del año.
\end{itemize}

Al ser estas metricas (GNI, deflactor, población) ya calculadas por tener interes en sí mismas, no supone una dificultad adicional obtener esta metrica derivada. Adicionalmente, es una metodología confiable ya que su confianza se deriva de la confianza que tenemos en la capacidad de medir las metricas base.

Además, las metricas base que se utilizán para calcular esta metrica tienen la ventaja adicional de ser valores aditivos. Es decir, cada uno de estos puede ser medido de forma independiente en subdivisiones de la economía a estudiar (por ejemplo las provincias de un país, y dentro de estas los municipios) y luego simplemente sumar las cantidades medidas por cada subdivisión para obtener la cantidad correspondiente al total de la economía. 

A su vez, cómo el GNI se pueden descomponer en distintos componentes, estos componentes pueden calcularse independientemente.

Se puede hacer esto utilizando que $GNI = GDP + EX_{net}$, donde:
\begin{itemize}
    \item GDP (Gross Domestic Product, Producto Interior Bruto) es la producción total de bienes y servicios en el país
    \item $EX_{net}$ es el neto de pagos y transferencisa (salvo importaciones y exportaciones, que se incluyen en el GDP) hacia el exterior (donde el signo positivo índica que a la economía ingreso más dinero del que salió)
\end{itemize}

Y a su vez podemos descomponer al principal sumando, el GDP, en sus componentes de dos maneras distintas, utilizando la Perspectiva de los Gastos y la Perspectiva de los ingresos.

\textbf{Perspectiva de Gastos}

$$
GDP = C + I + G + (X-M)
$$

Donde:
\begin{itemize}
    \item C = consumo
    \item I = Inversión
    \item G = Gasto del gobierno
    \item X = Exportaciones
    \item M = Importaciones
\end{itemize}

\textbf{Persectiva de los ingresos}

$$
GDP = S\ +\ B\ +\ R\ +\ I\ + II\ +\ D\ -\ SU 
$$

Donde:
\begin{itemize}
    \item S = Salarios
    \item B = Beneficios empresariales
    \item R = Rentas (alquileres)
    \item I = Intereses
    \item II = Impuestos Indirectos (ejemplo: IVA)
    \item D = Depreciación de bienes de cápital
    \item SU = Subsidios
\end{itemize}

El tener estos dos enfoques permite:

\begin{itemize}
    \item Descomponer el calculo del GNI en muchas variables chicas que se pueden medir de forma especializada y tienen interes en si mismo. Por tanto, se vuelve en si mismo una estadistica derivada de otras.
    \item Al tener 2 metodos independientes de calcular el GDP, que es el factor más importante en el computo del GNI, es posible detectar y corregir errores e inconsistencias en las mediciones.
\end{itemize}

En sintesis, la medición actualmente utilizada supone una gran cantidad de ventajas en matería logistica y de ser consecuencia de otra bateria de mediciones con interes en si mismas.

\subsection{Desventajas}

Hay un primer problema que podemos observar en esta metodología y es que el GNI incluye también los ingresos empresariales y del gobierno, mientras que el $BE$ lo definimos a nivel de individuos y sus ingresos personales. Sin embargo, como las empresas y gobieros pueden utilizar estos recursos para proveer bienes y servicios a las personas (que a estas no se les imputan dentro de sus ingresos personales, por ejemplo la educación pública no arancelada), este alejamiento de la definición que dimos originalmente puede permitir capturar mejor el bienestar económico promedio de una sociedad. 

Sin embargo, hay otro problema que en principio es más importante a tener en cuenta. La utilidad marginal decreciente aplica en los ingresos de cada individuo y no en los ingresos agregados de la sociedad. Por tanto, es importante aplicar la función de utilidad ($U$, en teste caso $log$) a los ingresos de cada individuo y no al promedio de los ingresos. Y lo importante es que \textbf{el logaritmo del promedio no es el promedio de los logaritmos}

Es esta última desventaja la que este trabajo busca subsanar proponiendo una medición alternativa, y a su vez evaluar que mejoras ofrece dicha alternativa frente a la metodogía actual.

\section{Propuesta de mejora: Medición ideal}

Dada una población $X$ de $N$ individuos cuyos ingresos son $X_1, X_2, \dots, X_N$, definimos el Bienestar Económico (promedio) de esa población como:

$$
    BE(X) = \frac{1}{N} * \sum_{i=1}^N log(X_i) = log(\sqrt[N]{\prod_{i=1}^N X_i})
$$

\subsection{Ventajas}

Es la medición exacta de la definición que propusimos de Bienestar Económico. Por tanto, es la mejor aproximación a dicho concepto. Lo que tiene especial sentido si pensamos que los ingresos son un instrumento para obtener bienestar y no el bienestar en si mismo, y eso es lo que se busca reflejar al aplicarles una función de utilidad distinta de la identidad.



Adicionalmente, al contemplar que un aumento en los ingresos de individuos que ya tienen altos ingresos les genera un menor beneficio marginal que aumentar en la misma cantidad absoluta los ingresos de individuos de menores ingresos, esta metrica refleja le eficiencia de la distribución del ingreso en una sociedad.



De esta forma, esta metrica premia simultaneamente un aumento de la productividad de una ecónomia como una distribución más eficiente de la misma.



\subsection{Desventajas}


\section{Propuesta de mejora: Mediciones aproximadas}

En esta sección introducimos el concepto de granularidad de una medición, y estudiamos soluciones de compromiso para crear una medición que se aproxima más a la que proponemos como medición ideal pero siendo logisticamente factibles.
Serán las mediciones que utilizaremos con los datos empiricos.

\chapter{Técnicas a utilizar}

En este cápitulo se explicarán brevemente las metricas y técnicas más importantes a utilizar en la presente tésis.

\section{Correlación de Pearson}

Coeficiente clásico de correlación líneal

\section{Correlaciones no lineales}

Coeficientes de correlación de Spearman y $\tau$ de Kendall.

\section{Bootstrap}

Técnica de remuestreo para estimar la distribución de los datos o de una función de los datos.

\chapter{Datos sinteticos}

En este cápitulo se utilizán datos simulados para explorar caracteristicas de los índicadores (existentes y propuestos) en sí mismos.

\section{Generación}
En esta sección se comentará qué datos sinteticos se generaron, que asunciones se hicieron sobre la distribución de los mismos, y un breve comentario sober las tecnologías utilizadas.

\section{Datos $LN(0,1)$}

En esta sección se analizán los datos generados utilizando que los ingresos de una población siguen una distribución $LogNormal(0,1)$. Se estudiará las distribuciones de las mediciones con distinta granularidad así como la correlación entre las mismas.

\section{Datos $LN(0,\sigma^2)$}

En esta sección se analizán los datos generados utilizando una distribución $LogNormal(0,\sigma^2)$ para distintos valores de $\sigma^2$. Trabajar con distintos valores de $\sigma^2$ permite una mayor variedad y riqueza de análisis.

\subsection{Comportamiento condicional}

Se estudia el comportamiento condicional a $\sigma^2$, tratandolo como 10 poblaciones independientes y estudiandolas.

\subsection{Interacción con granularidad}

Se estudia como se comportan las mediciones con distintas granularidades al modificar $\sigma^2$

\subsection{Comportamiento no condicional}

Se trabajará con toda la muestra sin condicionar por $\sigma^2$, es decir tratandola como una unica población heterogenea.

\chapter{Datos reales: Distribución de los ingresos}

En esta sección se utilizán datos reales de distribución del ingreso de distintos países y años para estudiar su distribución y compararlos con los datos sinteticos antes generados.

\section{Datos disponibles}

Se presenta la base de datos disponible. Se comentan los distintos alcances (nacional, urbano y rural) y tipos (ingresos y consumo) de las encuestas disponibles.

\section{Distribución empirica}

Se estudian las distribuciones empiricas disponibles y se las compara con los datos sinteticos generados en el cápitulo anterior.

\chapter{Datos reales: Mejora de la capacidad predictiva}

En este cápitulo exploramos si las modificaciones propuestas en la medición del bienestar económico mejoran la capacidad de predicir otros indicadores socio económicos.

\section{Esperanza de Vida al Nacer (EVN)}

Se estudia pormenorizadamente el caso de la EVN, discutiendo distintas formas de evaluar la capacidad predictiva de las distintas mediciones. Se utiliza el año como variable de control para evitar que una correlación entre el bienestar economico y el año y entre el año y la EVN aumente artificialmente la correlación entre el bienestar economico y la EVN.


\section{Otros indicadores}

Utitizando la metrica comparativa finalmente desarrollada para el caso de la EVN, se aplica la misma metrica a otros indicadores socio economicos (ej: Alfabetización a los 15 años) para ver cómo se comportan es estos casos.



\chapter{Conclusiones}

Titulo autoexplicativo

\chapter{Posibles trabajos futuros}

Titulo autoexplicativo

\chapter{Citas bibliograficas}

Titulo autoexplicativo

%%%% BIBLIOGRAFIA
\backmatter
\bibliographystyle{unsrt}
\bibliography{tesis}

\end{document}
