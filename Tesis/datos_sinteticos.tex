\chapter{Datos sintéticos} \label{chapter:datos_sinteticos}

En este capítulo se utilizan datos simulados para explorar características de los indicadores (existentes y propuestos) en sí mismos.

\section{Generación}

Para la generación de los datos se modela la distribución del ingreso en una sociedad como $logNorm(\mu,\sigma^2)$, basándonos en (Gibrat, 1931)\cite{gibrat1931inégalités}. Es decir, sea $X$ una población con $N$ individuos cuyos ingresos son $X_1, X_2, \dots, X_N$, las variables $X_i$ son IID (independientes e idénticamente distribuidas) y $X_i \sim LN(\mu, \sigma^2)$.

Recordar que 

$$
    X \sim LN(\mu, \sigma^2) \iff log(X) \sim N(\mu,\sigma^2)
$$

Notar que bajo esta distribución:

\begin{itemize}
    \item El bienestar económico tiene distribución normal $N(\mu,\sigma^2)$
    \item $BE_N(X)$ es un estimador insesgado de $\mu$
\end{itemize}

La metodología consistió en lo siguiente:

\begin{itemize}
    \item Generar poblaciones con $N = 1.000.000$ individuos cada una.
    \item Ordenar a los individuos de la población, en orden creciente de ingresos.
    \item Para cada población $X^i$ y para cada divisor $G$ de $N$, calcular $BE_G(X^i)$
    \item Almacenar para posterior uso los valores $BE_j^i = BE_{G_j}(X^i)$
\end{itemize}

Notar que todas las observaciones y todas las poblaciones generadas son independientes entre si.

Como modificar $\mu$ es lo mismo que multiplicar a todos los $X_i$ por $e^{\Delta \mu}$, utilizaremos $\mu = 0$ para la generación de datos sintéticos.

Respecto del valor de $\sigma^2$, se generaron 2 datasets:

\begin{itemize}
    \item \textbf{Datos $LN(0,1)$}: Un dataset donde $X_i \sim LN(0,1)$, para el que se simularon $20.000$ poblaciones.
    \item \textbf{Datos $LN(0,\sigma^2)$}: Un dataset donde se tomaron diversos valores de $\sigma^2$, en el cual se simularon $1.000$ poblaciones para cada valor entero de $\sigma^2$ entre 1 y 10, generando $10.000$ poblaciones en total.
\end{itemize}

La generación de datos se paralelizó utilizando GPU mediante CUDA\cite{lasorsa2024simluacion_cuda}.

\section{Datos $LN(0,1)$}

\begin{figure}[H] % 'h' significa que intenta colocar la figura aquí
    \centering % Centra la figura
    \includegraphics[width=\textwidth]{../figuras/figura_1_dispersion_plot_completo.png} % Cambia el nombre del archivo
    \caption{Distribución de $BE_j^i$, donde el eje X es la granularidad (cantidad de grupos) de la medición. Para facilitar la visualización, a cada elemento se le restó el mínimo de todo el dataset}
    \label{fig:1} % Etiqueta para referenciar
\end{figure}

Como puede verse en \ref{fig:1}, al tener más de 200 grupos es difícil distinguir las distintas distribuciones, inclusive en escala logarítmica. A su vez, se nota que cuando la granularidad es baja, cada aumento de granularidad acerca significativamente $BE_{G_j}$ a $BE_N$

Esto nos permite formular la primera de las hipótesis del presente trabajo:

\begin{hipotesis}[Mejora de la granularidad]\label{hipo:1}
    \\
    Sea $P$ un conjunto de poblaciones, donde cada $P_i \in P$ es una población de $N$ individuos $P_{i,j}$ $(1 \leq j \leq N)$, con $P_{i,j}$ variables IID y $P_{i,j} \sim LN(0,1)$, y sean $G_1, G_2$ tales que $G_1|N, G_2|N, G_1 < G_2$, entonces $\exists N_0 / \forall N > N_0 $, $Pearson(BE_{G_1},BE_N) < Pearson(BE_{G_2},BE_N)$ para casi toda población $P$.
    
    Para el intento de falsación utilizando datos sintéticos de esta hipótesis, es necesaria la hipótesis auxiliar de $N_0 \leq 1.000.000$    
%
\end{hipotesis}

Como puede verse en la figura \ref{fig:3}, la hipótesis \ref{hipo:1} resulta verosímil con los datos simulados con distribución $LN(0,1)$ 

\begin{figure}[H]
    \centering 
    \includegraphics[width=\textwidth]{../figuras/figura_3_mediciones_aproximadas_vs_ideal.png} % Cambia el nombre del archivo
    \caption{Aplicación de métricas de correlación lineal y no lineal entre $BE_G$ y $BE_N$ para todos los $G | N$ en muestras $LN(0,1)$ con $N=10^6$}
    \label{fig:3}
\end{figure}

De hecho, la figura \ref{fig:3} nos permite ser más ambiciosos y enunciar que aumentar la granularidad mejora la correlación tanto en el caso lineal (Pearson) como aplicando diversas métricas de correlación no lineal (Spearman y Kendall)

Notar que a su vez podemos aplicar la correlación de Spearman a la relación entre granularidad y las correlaciones (los ejes X e Y del gráfico), y para las 3 series da una correlación de 1 (dado que están perfectamente ordenados) y un p-valor de aproximadamente 0.

\section{Datos $LN(0,\sigma^2)$}

Al tener distintos valores de $\sigma^2$ podemos plantear todo un abanico de preguntas nuevas:

\begin{enumerate}
    \item \textbf{¿Cómo cambia la correlación entre $BE_G$ y $BE_N$ condicional a $\sigma^2$?}
    \item \textbf{¿Cómo es la correlación entre $BE_{G_i}$ y $BE_{G_j}$ condicional a $\sigma^2$?}
    \item \textbf{¿Cómo es la correlación entre $BE_G$ y $BE_N$ no condicional a $\sigma^2$?}
\end{enumerate}

\subsection{Comportamiento condicional}

Lo primero que podemos pretender es generalizar \ref{hipo:1} para todos los valores de $\sigma^2$:

\begin{hipotesis}[Mejora de la granularidad generalizada]\label{hipo:2}
    \\
    Sea $P^N$ un conjunto de poblaciones, donde cada $P^N_i \in P$ es una población de $N$ individuos $P^N_{i,j}$ $(1 \leq j \leq N)$, con $P^N_{i,j}$ variables IID y $P^N_{i,j} \sim LN(0,\sigma^2)$, y sean $G_1, G_2$ tales que $G_1|N, G_2|N, G_1 < G_2$, entonces $\exists N_0 / \forall N > N_0 $, $Pearson(BE^N_{G_1},BE^N_N) < Pearson(BE^N_{G_2},BE^N_N)$ para casi toda población $P^N$. Notar que el super índice $N$ indica el tamaño de las poblaciones. A su vez, como $G_1$ y $G_2$ son divisores de $N$, el conjunto $G$ al que pertenecen $G_1$ y $G_2$ es función de $N$, entonces es $G^N$.
    
    Para la falsación utilizando datos sintéticos de esta hipótesis, es necesaria la hipótesis auxiliar de $N_0 \leq 1.000.000$  
\end{hipotesis}

Notar que la hipótesis \ref{hipo:2} es una generalización no trivial de \ref{hipo:1} ya que modificar el $\sigma^2$ tiene efectos no lineales en la distribución.

Como puede verse en la figura \ref{fig:4}, esta hipótesis generalizada también resulta verosímil con los datos generados.

\begin{figure}[H]
    \centering 
    \includegraphics[width=\textwidth]{../figuras/figura_4_mediciones_aproximadas_vs_ideal_varianza.png} 
    \caption{Aplicación de métrica de correlación lineal entre $BE_G$ y $BE_N$ para todos los $G | N$ en muestras $LN(0,\sigma^2)$ con $N=10^6$, condicional a distintos valores de $\sigma^2$}
    \label{fig:4}
\end{figure}

Algo que podemos notar observando \ref{fig:4} es que para varianzas altas la correlación lineal entre las granularidades bajas y la medición ideal es notablemente baja.

Esto puede apreciarse mejor en \ref{fig:5} , donde podemos ver que mientras que $BE_1$ y $BE_N$ tienen una tendencia marcada para $\sigma^2=1$, para $\sigma^2=10$ se comportan como variables prácticamente independientes.

\begin{figure}[H]
    \centering 
    \includegraphics[width=\textwidth]{../figuras/figura_5_dist_conjuntas_var_y_cgrupos.png} 
    \caption{Dispersión conjunta de las métricas entre distintas granularidades, condicional a distintos valores de $\sigma^2$. Puede verse cómo la correlación entre las distintas granularidades (ejes X e Y de cada gráfico) disminuye al aumentar la varianza (indicada en el título de cada cuadro)}
    \label{fig:5}
\end{figure}



\subsection{Interacción con granularidad}

Lo observado en \ref{fig:4} y \ref{fig:5} nos puede llevar a proponer una hipótesis ambiciosa sobre la relación entre la varianza de la población y la granularidad de la medición:

\begin{hipotesis}[Relación entre granularidad y varianza]\label{hipo:3}
    \\

    Sean:

    \begin{itemize}
    \item $P^N$ un conjunto de poblaciones, donde cada $P^N_i \in P$ es una población de $N$ individuos $P^N_{i,j}$ $(1 \leq j \leq N)$, con $P^N_{i,j}$ variables IID y $P^N_{i,j} \sim LN(0,\sigma_P^2)$ 
    \item $Q^N$ un conjunto de poblaciones, donde cada $Q_i \in P$ es una población de $N$ individuos $Q^N_{i,j}$ $(1 \leq j \leq N)$, con $Q_{i,j}$ variables IID y $Q^N_{i,j} \sim LN(0,\sigma_{Q^N}^2)$, con $\sigma_{Q^N}^2 > \sigma_{P^N}^2$
    \item $G_1, G_2 \in Divs(N)$, con $G_1 \neq G_2$.
    \end{itemize}

    
    Entonces, $\exists N_0/ \forall N > N_0, $ $Pearson_{P^N}(BE^N_{G_1},BE^N_{G_2}) > Pearson_{Q^N}(BE^N_{G_1}, BE^N_{G_2})$, $\forall G_1, G_2 (G_1 \neq G_2)$ y para casi toda tupla $(P^N,Q^N)$, notar que el super índice $N$ indica el tamaño de las poblaciones. A su vez, como $G_1$ y $G_2$ son divisores de $N$, el conjunto $G$ al que pertenecen $G_1$ y $G_2$ es función de $N$, entonces es $G^N$.
    

    Para la falsación utilizando datos sintéticos de esta hipótesis, es necesaria la hipótesis auxiliar de $N_0 \leq 1.000.000$
\end{hipotesis}

Sorprendentemente, la hipótesis \ref{hipo:3} resulta ser inverosímil con los datos, como puede verse en la figura \ref{fig:12}, y también lo es si cambiamos $Pearson_P(BE_{G_1},BE_{G_2}) > Pearson_Q(BE_{G_1}, BE_{G_2})$ por $Spearman_P(BE_{G_1},BE_{G_2}) > Spearman_Q(BE_{G_1}, BE_{G_2})$, como puede verse en \ref{fig:13}

\begin{figure}[H]
    \centering 
    \includegraphics[width=\textwidth]{../figuras/figura_12_pearson_hasta_100_grupos.png} 
    \caption{En los datos simulados con diversas varianzas, la $Pearson(BE_i,BE_{100} | \sigma^2)$, donde cada serie representa un nivel de granularidad y el eje X los distintos $\sigma^2$}
    \label{fig:12}
\end{figure}

Observando \ref{fig:12}  y \ref{fig:13} podemos también notar que:

\begin{itemize}
    \item Para un mismo valor de $\sigma^2$, aumentar la granularidad siempre mejora la correlación con el caso en el que tomamos 100 grupos. Esto es verosímil con \ref{hipo:2}
    \item Las series tienen un comportamiento similar, no en nivel pero si en cambio, frente a los cambios en $\sigma^2$
\end{itemize}

\begin{figure}[H]
    \centering 
    \includegraphics[width=\textwidth]{../figuras/figura_13_spearman_hasta_100_grupos.png} 
    \caption{En los datos simulados con diversas varianzas, la $Spearman(BE_i,BE_{100} | \sigma^2)$, donde cada serie representa un nivel de granularidad y el eje X los distintos $\sigma^2$}
    \label{fig:13}
\end{figure}



\subsection{Comportamiento no condicional}

Lo estudiado en \ref{hipo:2} nos lleva a preguntarnos cómo se comporta la correlación entre granularidades cuando tenemos diferentes $\sigma^2$ coexistiendo en la misma muestra.

Lo observado en \ref{fig:15} nos indica que tener poblaciones con distintas $\sigma^2$ aumenta enormemente la correlación entre las distintas granularidades.

\begin{figure}[H]
    \centering 
    \includegraphics[width=\textwidth]{../figuras/figura_15_simulados_vs_100.png} 
    \caption{En los datos simulados con diversas varianzas, $Spearman(BE_i,BE_{100})$, donde el eje X son las distintas granularidades (cantidad de grupos). Notar que el eje Y está en el rango [0.99,1]. La serie \textbf{simulados estandarizados} corresponde a, para una población $P_i$, restarle a cada $BE_{G_j}(P_i)$ el valor de $BE_N(P_i)$, para evitar que medias poblacionales ligeramente distintas (en los logaritmos) influyan en la correlación. Lo notable es que hacer esto aumenta la correlación en vez de reducirla.}
    \label{fig:15}
\end{figure}

Esta característica lleva a pensar que en el caso empírico, como no tenemos a lo países agrupados por $\sigma^2$, va a haber una correlación fuerte entre las distintas granularidades. 