%\begin{center}
%\large \bf \runtitle
%\end{center}
%\vspace{1cm}
\chapter*{\runtitle}

The use of the logarithm of the average income as an indicator of the economic well-being of a society has several problems. This work proposes a new indicator, the average of logarithms of income, as a better measurement of economic well-being. We verified that, if income is distributed in a log-normal manner, this can be approximated with existing income surveys. At the same time, we compare whether it results in a better predictor of other variables related to economic well-being than the original methodology and other alternative measurements, correcting for inequality or taking the logarithm of the median, arriving at the conclusion that these alternative measurements behave in a way that is indistinguishable from the measurement proposed in this work, and that in turn there is a significant number of cases where the best predictor is the logarithm of the average income. This was verified for both income surveys and consumption surveys (using consumption as a proxy for income).
\bigskip

\noindent\textbf{Keywords:} Income, Economic well-being, Measurement, Correlation, Life Expectancy, Consumption.