%\begin{center}
%\large \bf \runtitulo
%\end{center}
%\vspace{1cm}
\chapter*{\runtitulo}

El uso del logaritmo del ingreso promedio como indicador del bienestar ecónomico de una sociedad tiene diversos problemas. El presente trabajo propone un nuevo indicador, el promedio de los logaritmos de los ingresos, como mejor medición del bienestar economico. Comprobamos que, si los ingresos distribuyen e forma log-normal, esta puede ser aproximada con las encuestas de ingresos ya existentes. A su vez, comparamos si resulta en un mejor predictor de otras variables relacionadas al bienestar economico que la metodología original y que otras mediciones alternativas, corregir por desigualdas o tomar el logaritmo de la mediana, llegando a la conclusión de que estas mediciones alternativas se comportan de forma indistinguible a la medición propuesta en este trabajo, y que a su vez existe una cantidad significativa de casos donde el mejor predictor es el logaritmo del promedio de los ingresos. Esto se comprobo tanto para encuestas de ingresos como para incuestas de consumo (usando el consumo como proxy del ingreso)

\noindent\textbf{Palabras claves:} Ingresos, Bienestar económico, Medición, Correlación, Esperanza de Vida, Consumo.