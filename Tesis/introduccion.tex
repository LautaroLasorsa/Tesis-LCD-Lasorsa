\chapter{Introducción}

Llamemos bienestar económico a la utilidad que los individuos obtienen de sus ingresos. Actualmente, para estimar el bienestar económico promedio de una sociedad se calcula el ingreso total de esa sociedad (obtenido por individuos, empresas y gobiernos), se lo corrige por el nivel de precios del país, se lo divide por la cantidad de habitantes y se calcula la utilidad que un individuo obtendría de tener esa cantidad de ingresos. El problema que tiene este indicador es que la función de utilidad no es lineal, por tanto no es lo mismo la utilidad que un individuo obtendría de tener el ingreso promedio que el promedio de la utilidad que los individuos obtienen de sus ingresos. 

Existen alternativas a esta indicador, una es corregir este indicador por la desigualdad, que tiene el problema de que la penalizan de forma artificial y no como un resultado natural de las ineficiencias económicas que genera. Otras alternativas, como la mediana o la proporción de la población que esta debajo de cierto umbral, tienen el problema de sintetizar excesivamente la información y ser insensibles a grandes partes de la distribución.

En el presente trabajo, estudiamos si al estimar primero la utilidad que cada individuo obtiene de sus ingresos y luego tomar el promedio de estas utilidades obtenemos un mejor estimador, el cómo nos podemos aproximar a este indicador ideal con los datos disponibles y si nos permite predecir mejor otros indicadores socioeconómicos.

En el capítulo \ref{chapter:mediciones_bienestar_economico} se comparará la metodología actual para medir el bienestar económico de una sociedad y se propondrá una medición alternativa. En el capítulo \ref{chapter:datos_sinteticos} se buscará aplicar ambos métodos a un conjunto de datos sintéticos para comparar su comportamiento. En el capítulo \ref{chapter:datos_reales_distribucion} se comenta la disponibilidad de datos reales sobre distribución del ingreso, y en el capítulo \ref{chapter:datos_reales_otros_indicadores} se busca comparar ambos indicadores (propuesto y actual) mediante su capacidad de predecir otras variables de interés, como la esperanza de vida al nacer. El capítulo \ref{chapter:tecnicas_a_utilizar} explica las distintas técnicas que se utilizarán en este trabajo, para quienes no las conozcan o quieran refrescarlas antes de seguir la lectura, y los últimos 2 capítulos son sobre las conclusiones y posibles trabajos futuros.

