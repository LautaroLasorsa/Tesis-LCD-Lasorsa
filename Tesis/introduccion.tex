\chapter{Introducción}

Llamemos bienestar ecónomico a la utilidad que los individuos obtienen de sus ingresos. Actualmente, para estimar el bienestar ecónomico promedio de una sociedad se calcula el ingreso total de esa sociedad (obtenido por individuos, empresas y gobiernos), se lo corrige por el nivel de precios del país, se lo divide por la cantidad de habitantes y se calcula la utilidad que un individuo obtendría de tener esa cantidad de ingresos. El problema que tiene este indicador es que la función de utilidad no es líneal, por tanto no es lo mismo la utilidad que un individuo obtendría de tener el ingreso promedio que el promedio de la utilidad que los individuos obtienen de sus ingresos. 

Alternativas como corregir este indicador por la desigualdad tienen el problema de que la penalizán de forma artificial y no como un resultado natural de las ineficiencias economicas que genera. Otras alternativas, como la mediana o la proporción de la población que está debajo de un cierto umbral, tienen el problema de sintetizar excesivamente la información y ser insensibles a grandes partes de la distribución.

En el presente trabajo, estudiamos si el primero estimar la utilidad que cada individuo obtiene de sus ingresos y luego tomar el promedio de estas utilidades resulta en un mejor estimador, el cómo nos podemos aproximar a este indicador ideal con los datos disponibles y si nos permite predecir mejor otros indicadores socioeconomicos.

En el cápitulo \ref{chapter:mediciones_bienestar_economico} se comparará la metodología actual para medir el bienestar ecónomico de una sociedad y se propondrá una medición alternativa. En el cápitulo \ref{chapter:datos_sinteticos} se buscará aplicar ambos metodos a un conjunto de datos sinteticos para comparar su comportamiento. En el cápitulo \ref{chapter:datos_reales_distribucion} se comenta la disponibilidad de datos reales sobre distribución del ingreso, y en el cápitulo \ref{chapter:datos_reales_otros_indicadores} se busca comparar ambos indicadores (propuesto y actual) mediante su capacidad de predecir otras variables de interes, como la esperanza de vida al nacer. El cápitulo \ref{chapter:tecnicas_a_utilizar} explica las distintas técnicas que se utilizarán en este trabajo, para quienes no las conozcan o quieran refrescarlas antes de seguir la lectura, y los últimos 2 cápitulos son sobre las conclusiones y posibles trabajos futuros.

