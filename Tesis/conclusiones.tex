\chapter{Conclusiones} \label{chapter:conclusiones}

\subsection{Conclusión principal}

\textbf{La principal conclusión del presente trabajo es que el indicador propuesto puede ser aproximado satisfactoriamente con los datos disponibles, pero el mismo no captura mejor el concepto de bienestar económico que otras metodologías ya existentes.}

Si asumimos que los datos se distribuyen de forma log-normal, esto se puede atribuir a que una de las otras metodologías (tomar la mediana de los ingresos) es también un estimador insesgado de la media, moda y mediana (el mismo valor) de la normal asociada.

\subsection{Otras conclusiones}

De lo trabajado en esta tesis también podemos decir:

\begin{itemize}
    \item El uso discriminado de encuestas de ingresos y de consumo permitió distinguir comportamientos diferentes. Esto lleva a presumir que las condiciones que llevan a un país a utilizar uno u otro tipo de encuesta también influyen en los fenómenos estudiados, y/o que son dos metodologías de muy diferente calidad.
    
    \item El uso de correlaciones no lineales no permitió encontrar patrones o comportamientos no observados al utilizar la correlación de Pearson. Esto puede verse en  \ref{fig:8},\ref{fig:9}, \ref{fig:18}, \ref{fig:21}, \ref{fig:22} y \ref{fig:23} donde al utilizar correlaciones no lineales no pudimos observar patrones no observados al utilizar la correlación de Pearson.
    
    \item El uso de bootstrap no paramétrico no permitió, tampoco, observar patrones o comportamientos no observados originalmente. Esto puede observarse en \ref{fig:7}, \ref{fig:8}, \ref{fig:9}, \ref{fig:20} y \ref{fig:21}, donde el rango intercuartílico estimado por bootstrap se comporto de forma paralela al dato observado originalmente.

    \item La fuerte correlación entre $BE_1$ y $BE_N$ tanto en los datos simulados, cuando hay múltiples varianzas coexistiendo, como en los empíricos es una explicación parcial al hecho de que el indicador propuesto no fuera superador de lo ya existente.
\end{itemize}

\subsection{Falsación de las hipótesis}

Las hipótesis presentadas en este trabajo fueron:

\begin{enumerate}
    \item \ref{hipo:1} y \ref{hipo:2}, que dicen que mejorar la granularidad de $BE_i$ aumenta la correlación entre $BE_i$ y $BE_N$. Los resultados observados en los datos son verosímiles con estas hipótesis.
    \item \ref{hipo:3}, que dice que al aumentar la varianza, la correlación entre $BE_i$ y $BE_j$ ($i \neq j$) se reduce. Los resultados observados no son verosímiles con esta hipótesis.
    \item \ref{hipo:4}, que dice que la medida propuesta es mejor predictor de otras variables relacionadas al bienestar económico que las medidas existentes. Los resultados observados no son verosímiles con esta hipótesis.
\end{enumerate}

