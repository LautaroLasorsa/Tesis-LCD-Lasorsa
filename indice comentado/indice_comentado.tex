\nonstopmode 
%\documentclass[11pt,a4paper,twoside]{tesis}
% SI NO PENSAS IMPRIMIRLO EN FORMATO LIBRO PODES USAR
\documentclass[11pt,a4paper]{tesis}
\usepackage{titlesec}
\titleformat{\chapter}[hang]{\normalfont\huge\bfseries}{\thechapter}{1em}{}
\titlespacing*{\chapter}{0pt}{0pt}{20pt}

\makeatletter
\renewcommand\chapter{\if@openright\else\fi
  \thispagestyle{plain}%
  \global\@topnum\z@
  \@afterindentfalse
  \secdef\@chapter\@schapter}
\makeatother

\usepackage{graphicx}
\usepackage[utf8]{inputenc}
\usepackage[spanish]{babel}
\usepackage[left=3cm,right=3cm,bottom=3.5cm,top=3.5cm]{geometry}

\begin{document}

%%%% CARATULA

\def\autor{Lautaro Lasorsa}
\def\tituloTesis{Propuesta de mejora en la medición del bienestar económico}
\def\runtitulo{Propuesta de mejora en la medición del bienestar económico}
\def\director{Rodrigo Castro}
\def\codirector{Walter Sosa Escudero}
\def\lugar{Buenos Aires, Argentina, 2024}
\input{caratula}

%%%% ABSTRACTS, AGRADECIMIENTOS Y DEDICATORIA
% \frontmatter

% \tableofcontents

% \mainmatter
% \pagestyle{headings}

%%%% ACA VA EL CONTENIDO DE LA TESIS

\chapter{Medición del Bienestar Ecónomico}

En este cápitulo se presenta la problematica a tratar en la tésis, se motiva su estudio y se comparán las formas actuales de medirlo con las alternativas propuestas.

\section{Definición y motivación}

En esta sección definimos qué es el bienestar económico y otros conceptos relacionados y por qué es importante medirlo bien.

\section{Medición actual}

En esta sección comentamos la forma actualmente utilizada, entre otros por quienes elaboran IDH, para medir el bienestar económico promedio de una población, analizando sus ventajas y desventajas.

\section{Propuesta de mejora: Medición ideal}

En esta sección comentaremos cuál es la que proponemos como la forma ideal de medir el bienestar económico, la justificación de esta propuesta y sus límitaciones, sobre todo a nivel logistico.

\section{Propuesta de mejora: Mediciones aproximadas}

En esta sección introducimos el concepto de granularidad de una medición, y estudiamos soluciones de compromiso para crear una medición que se aproxima más a la que proponemos como medición ideal pero siendo logisticamente factibles.
Serán las mediciones que utilizaremos con los datos empiricos.

\chapter{Técnicas a utilizar}

En este cápitulo se explicarán brevemente las metricas y técnicas más importantes a utilizar en la presente tésis.

\section{Correlación de Pearson}

Coeficiente clásico de correlación líneal

\section{Correlaciones no lineales}

Coeficientes de correlación de Spearman y $\tau$ de Kendall.

\section{Bootstrap}

Técnica de remuestreo para estimar la distribución de los datos o de una función de los datos.

\chapter{Datos sinteticos}

En este cápitulo se utilizán datos simulados para explorar caracteristicas de los índicadores (existentes y propuestos) en sí mismos.

\section{Generación}
En esta sección se comentará qué datos sinteticos se generaron, que asunciones se hicieron sobre la distribución de los mismos, y un breve comentario sober las tecnologías utilizadas.

\section{Datos $LN(0,1)$}

En esta sección se analizán los datos generados utilizando que los ingresos de una población siguen una distribución $LogNormal(0,1)$. Se estudiará las distribuciones de las mediciones con distinta granularidad así como la correlación entre las mismas.

\section{Datos $LN(0,\sigma^2)$}

En esta sección se analizán los datos generados utilizando una distribución $LogNormal(0,\sigma^2)$ para distintos valores de $\sigma^2$. Trabajar con distintos valores de $\sigma^2$ permite una mayor variedad y riqueza de análisis.

\subsection{Comportamiento condicional}

Se estudia el comportamiento condicional a $\sigma^2$, tratandolo como 10 poblaciones independientes y estudiandolas.

\subsection{Interacción con granularidad}

Se estudia como se comportan las mediciones con distintas granularidades al modificar $\sigma^2$

\subsection{Comportamiento no condicional}

Se trabajará con toda la muestra sin condicionar por $\sigma^2$, es decir tratandola como una unica población heterogenea.

\chapter{Datos reales: Distribución de los ingresos}

En esta sección se utilizán datos reales de distribución del ingreso de distintos países y años para estudiar su distribución y compararlos con los datos sinteticos antes generados.

\section{Datos disponibles}

Se presenta la base de datos disponible. Se comentan los distintos alcances (nacional, urbano y rural) y tipos (ingresos y consumo) de las encuestas disponibles.

\section{Distribución empirica}

Se estudian las distribuciones empiricas disponibles y se las compara con los datos sinteticos generados en el cápitulo anterior.

\chapter{Datos reales: Mejora de la capacidad predictiva}

En este cápitulo exploramos si las modificaciones propuestas en la medición del bienestar económico mejoran la capacidad de predicir otros indicadores socio económicos.

\section{Esperanza de Vida al Nacer (EVN)}

Se estudia pormenorizadamente el caso de la EVN, discutiendo distintas formas de evaluar la capacidad predictiva de las distintas mediciones. Se utiliza el año como variable de control para evitar que una correlación entre el bienestar economico y el año y entre el año y la EVN aumente artificialmente la correlación entre el bienestar economico y la EVN.


\section{Otros indicadores}

Utitizando la metrica comparativa finalmente desarrollada para el caso de la EVN, se aplica la misma metrica a otros indicadores socio economicos (ej: Alfabetización a los 15 años) para ver cómo se comportan es estos casos.



\chapter{Conclusiones}

Titulo autoexplicativo

\chapter{Posibles trabajos futuros}

Titulo autoexplicativo

\chapter{Citas bibliograficas}

Titulo autoexplicativo



\end{document}